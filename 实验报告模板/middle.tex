\fancypage{\fbox}{}
\subsubsection{幅频特性}
根据上述$U_R,U_C$的表达式,当$\omega \rightarrow 0$时,$U_R\rightarrow 0,U_C\rightarrow U$;$U_R$随着$\omega$增大逐渐增大,$U_C$反之;当$\omega \rightarrow \infty$时,$U_R\rightarrow U,U_C\rightarrow 0$。
\subsubsection{相频特性}
当$\omega$很低时,$\varphi_R→+\pi/2$;当$\omega$很高时,$\varphi_R→0$,且$\varphi_C=|\varphi_R|-\pi/2$
\subsubsection{等幅频率}
\begin{wrapfigure}[6]{r}{0.3\textwidth} %比如{r}{0.5\textwidth}
	\centering
	\includegraphics[height=3cm,width=5cm]{figure/2.png}
    \caption*{RL串联电路}
\end{wrapfigure}
当$R=1/(\omega C)$时,$U_R=U_C$,此时的频率为等幅频率,也叫截止频率
\subsection{RL电路的幅频特性}
RL的阻抗为$\tilde{Z}=R+j\omega L$,其模为$Z=|\tilde{Z}|=\sqrt{R^2+(\omega L)^2}$其他同RC电路
\subsection{RLC串联电路}
\begin{wrapfigure}[6]{r}{0.3\textwidth} %比如{r}{0.5\textwidth}
	\centering
	\includegraphics[height=3cm,width=5cm]{figure/3.png}
    \caption*{RLC串联电路}
\end{wrapfigure}
该电路的总阻抗为$\tilde{Z}=R+j\left(\omega L-\dfrac{1}{\omega C}\right)$

幅角:$\varphi=\arctan \dfrac{\omega L-\frac 1 {\omega C}}{R}$

R上的电压为$U_R=\dfrac{U}{Z}R $
\subsubsection{谐振频率}
当$\omega L-\dfrac{1}{\omega C}=0$时,$\varphi=0$,并且$U_R=U$为极大值,此时的频率记为谐振频率$f_0=\dfrac{1}{2\pi\sqrt{LC}}$
\subsubsection{相频特性}
$\omega <\omega _0 $时,此时电路呈电容性;$\omega >\omega _0 $时,此时电路呈电感性;$\omega =\omega _0 $时,此时电路呈电阻性。
\section{实验内容与步骤}
\subsection{测量并绘制RC串联电路的幅频、相频曲线}
\begin{enumerate}[(1)]
    \item 连接电路,接通各个仪器电源进行预热
    \item 调节信号源的$f=500Hz,U=3.0V_{RMS}=8.5Vpp$
    \item 依次从电压表上测出R,C上的电压U,U,.从示波器的李萨如图形上读出x轴与  图形相交的水平距离$2x_0$和图形在x轴上的投影$2X$
    \item 依次测出表格中其余$f$值条件下的$U_R,U_C$,和$\varphi$值.
\end{enumerate}
\subsection{测量并绘制RL串联电路的幅频、相频曲线}
与上一步内容相似,将C替换为L
\subsection{测量并绘制RLC串联电路的相频曲线}
其测量电路与以上内容相仿,只是将串联LC代替原来的C即可
\begin{enumerate}[(1)]
    \item 用李萨如图形找出谐振频率
    \item 测出f=350,600,700,780,900,1500Hz条件下的$\varphi$值
\end{enumerate}
\section{数据表格}
\begin{table}[h]
    \centering
    \caption*{RC幅频,相频曲线}
    \begin{tabular}[\linewidth]{|m{2cm}|m{2cm}|m{2cm}|m{2cm}|m{2cm}|m{2cm}|m{2cm}|}
        \hline
        \multicolumn{7}{|c|}{$U=3.0V_{RMS}=8.5Vpp$  $R=200\Omega $  $C=0.47\mu F$}\\
        \hline
        $f/Hz$ & 500 & 1200 & 1700 & 2000 & 3000 & 7000 \\
        \hline
        $U_R/V$ & & & & & & \\
        \hline
        $U_C/V$ & & & & & & \\
        \hline
        $2x_0/cm$ & & & & & & \\
        \hline
        $2X/cm$ & & & & & & \\
        \hline
        $\varphi/(^{\circ} )$ & & & & & & \\
        \hline
    \end{tabular}
    \caption*{RL幅频,相频曲线}
    \begin{tabular}[\linewidth]{|m{2cm}|m{2cm}|m{2cm}|m{2cm}|m{2cm}|m{2cm}|m{2cm}|}
        \hline
        \multicolumn{7}{|c|}{$U=3.0V_{RMS}=8.5Vpp$  $R=1000\Omega $  $L=0.1H$}\\
        \hline
        $f/Hz$ & 500 & 1200 & 1700 & 2000 & 3000 & 7000 \\
        \hline
        $U_R/V$ & & & & & & \\
        \hline
        $U_L/V$ & & & & & & \\
        \hline
        $2x_0/cm$ & & & & & & \\
        \hline
        $2X/cm$ & & & & & & \\
        \hline
        $\varphi/(^{\circ} )$ & & & & & & \\
        \hline
    \end{tabular}
\end{table}
